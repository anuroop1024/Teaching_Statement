\documentclass[12pt,a4paper,sans]{moderncv} % Font sizes: 10, 11, or 12; paper sizes: a4paper, letterpaper, a5paper, legalpaper, executivepaper or landscape; font families: sans or roman


\moderncvstyle{banking} % CV theme - options include: 'casual' (default), 'classic', 'oldstyle' and 'banking'
\moderncvcolor{blue} % CV color - options include: 'blue' (default), 'orange', 'green', 'red', 'purple', 'grey' and 'black'

\setlength{\parskip}{3em}
\usepackage{tgtermes}
\usepackage{setspace}
\usepackage{lettrine}
\usepackage[a4paper,left=2.7cm,right=2.3cm,top=2.1cm,bottom=2.5cm]{geometry}
\onehalfspacing
\firstname{} % Your first name
\familyname{} % Your last name

% All information in this block is optional, comment out any lines you don't need
\title{Teaching Statement}
\address{Anuroop Gaddam}{}
\mobile{(+64) 21 046 8363}
\email{anuroop24@gmail.com}

%\extrainfo{additional information}
%\photo[70pt][0.4pt]{pictures/picture} % The first bracket is the picture height, the second is the thickness of the frame around the picture (0pt for no frame)
%\quote{"A witty and playful quotation" - John Smith}

%----------------------------------------------------------------------------------------

\begin{document}
\makecvtitle % Print the CV title


\section{Teaching Philosophy}
The opportunity to teach (and learn) from students is one of the key reasons why I aspire to pursue a career in academia. Throughout my life, my teachers have been the foremost inspiration for my academic success, and it has since been my aspiration to similarly serve as a constructive influence on my students. Teaching, to me, is not just about being prepared and organised with the material, eloquently delivering lectures and making yourself available outside class, but it is also about demonstrating to students your genuine passion for the subject, thereby kindling their interest. I always employ the following methods (the 4-Es) to improve the holistic learning experience in my class:
\begin{enumerate}

\item \textbf{Explain context before concept:} \par Students learn better if they understand why learn what they learn. I will provide practical applications to various concepts and their significance in real world, making the class both exciting and relevant.
\item \textbf{Emphasise learning not scoring:}\par As a teacher, I will impress upon the students that my objective for them is to internalise the material and not to feel pressured by exams or approach them as a competition. This gives slow-learners the confidence to learn at their pace, promoting inclusive growth in class.
\item \textbf{Evaluate and re-orient:} \par As each class is unique, it is important to proactively refocus the course material and teaching style based on self-evaluation and feedback from students. 
\item \textbf{Encourage open-ended projects:} \par Given the limited time schedule,
it is naturally difficult to discuss all aspects of the topic in class and hence, there is always more to learn
beyond what is taught in the classroom. Also, with the wealth of resources available online and elsewhere,
it is key to give students the opportunity to explore and collaborate through relevant open-ended projects.
\end{enumerate}

%----------------------------------------------------------------------------------------
Through these principles, I strongly believe that I would be able to effectively deliver the course content and
exceed expectations as a course instructor. I also think teaching would reinforce my understanding of the
subject and my interaction with a diverse set of students can lead to interesting research directions.
%----------------------------------------------------------------------------------------
\section{Teaching Interests}
As such, I am eager to teach both undergraduate and graduate level courses related
to electronics and electrical engineering. Specifically, I am well equipped to instruct introductory and advanced
courses on analogue circuits and systems, logic design, circuit theory, engineering programming, electronics design automation, embedded systems, digital integrated circuits and signal processing. I am also keen to develop new graduate-level courses in the above areas that highlight the recent trends in technology like IoT and foster early research ideas. 

%----------------------------------------------------------------------------------------
\section{Teaching Experience}
During my PhD study at Massey University, I was  employed as teaching assistant for 281.335 Instrumentation Electronics and Control Engineering and 143.339 Design for Computer and Communication Systems. My experience as an teaching assistant reinforced my decision to pursue a career with opportunities for teaching.  Later as a Lecturer at Waikato Institute of Technology in New Zealand, I played a major role in introduction of Electronics Engineering major. I have to plan and prepare effective teaching resources, course descriptors lesson plans with learning outcomes, blended delivery methods and also program co-ordination. I was widely recognised for development and implementation of effective teaching strategies, facilitating blended learning, project based programmes and promoting learning that is consistent with student needs and industrial standards. I was instrumental in getting research projects for students from local industries to conduct R&D activities and which led to full employment of my students by the company, as well as letters of recognition for proficiently designed R&D processes. 

In summary, my academic experiences thus far have left me with a strong interest in teaching. I look
forward to an opportunity to continue my journey, both growing as a teacher and helping and benefiting
from the next generation.

\end{document}
